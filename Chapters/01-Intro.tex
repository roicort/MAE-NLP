\chapter[Introducción]{Introducción}
\label{cp:introduction}

{
\parindent0pt

\textit{Author: José Areia}

\textit{Current Version: 2.2.10}

\textit{License: \LaTeX~Project Public License v1.3c}

\textit{Official Repository: \href{https://github.com/joseareia/ipleiria-thesis}{GitHub Repository}}

\vspace{.935em}

Welcome to the \textcolor{maincolor}{\textit{IPLeiria Thesis}} template! Thank you for choosing it for your dissertation, report, or project. This template reflects many hours of development, and I hope you enjoy using it as much as I did creating it. This chapter introduces its purpose and helps you get started. See \autoref{cp:user-guide} for a detailed guide, and \autoref{cp:latex-tutorial} for a brief \LaTeX~tutorial to maximise its use.
}

\section{Motivation}
I've been using \LaTeX~since 2020 for a wide range of purposes. Over time, I've reviewed over a hundred templates, and there's always something missing. \textit{Always}. Powerful templates---\textit{i.e.}, highly customisable with many options---are often poorly organised. Well-organised ones usually lack flexibility. Some even compile with errors and warnings. Most importantly, many aren't user-friendly. So, I created my own template for theses and reports, tailored to the Polytechnic University of Leiria. My goals were: \(i)\) clear and structured file organisation, \(ii)\) a clean, professional, and attractive design, \(iii)\) high customisability, and \(iv)\) ease of use, especially for beginners.
